\documentclass[11pt,letterpaper]{article}

\usepackage[utf8]{inputenc}
\usepackage[T1]{fontenc}
\usepackage{microtype}
\usepackage{newspaper}

%
% Fonts
%

\usepackage{yfonts} % For Frakforter font used in the newspaper banner
\usepackage{amssymb} % in order to have the tombstone (black square) at the end of the article

% Those could help find new version/sizes of gothic fonts
% Especoally for gothic paragraph initials
%\DeclareFontFamily{U}{yfrak}{}
%\DeclareFontShape{U}{yfrak}{m}{n}{<-> yfrak}{}
%\DeclareFontShape{LYG}{ygoth}{m}{n}{<-> ygoth}{}

% Enable gothic initials
%\usepackage{lettrine} % if you want gothic as the first letter
%\renewcommand{\LettrineFontHook}{\usefont{U}{yinit}{m}{n}} % if you want to use yinit gothic font
% below would be settings for using with gothic initial, dont work well
%\setcounter{DefaultLines}{4}
%\setlength{\DefaultFindent}{0.5em}
%\setlength{\DefaultNindent}{0em}
%\renewcommand{\DefaultLoversize}{-0.65}


% Metadata
\date{\today}
\currentvolume{3}
\currentissue{1}

%%   (The colon after s: is to get a more
%%   modern majuscule s in this font instead of 
%%   the medieval tall s.

\SetPaperName{%
%  Univers:ity Times:%
  Divers\/ity Times%
}

%% The name used in the running header after
%% the first page
\SetHeaderName{Diversity Statement}

%% and also...
\SetPaperLocation{Washington DC}
\SetPaperSlogan{``All the News I Feel Like Printing.''}
\SetPaperPrice{Lubbock, TX}

% [LianTze] times (the package not the font) is rather outdated now; use newtx (see later)
% \usepackage{times}
\usepackage{graphicx}
\usepackage{multicol}

\usepackage{picinpar}
%uasage of picinpar:
%\begin{window}[1,l,\includegraphics{},caption]xxxxx\end{window}

%% [LianTze] Contains some modifications
\usepackage{newspaper-mod}
%%... so now you can redefine the headline and byline style if you want to.
%% These can be issued just before any
%% byline or headline in the paper, to
%% individually style each article
%%
% \renewcommand{\headlinestyle}{\itshape\Large\lsstyle}
% \renewcommand{\bylinestyle}{\bfseries\Large\raggedright}

\makeatletter
%%%%%%%%%%% Redefine \maketitle     %%%%%%%
\renewcommand{\maketitle}{\thispagestyle{empty}
\vspace*{-45pt}
\begin{center}
{\textgoth{  \fontencoding{T1}\fontfamily{yfrak}\fontsize{75}{0}\bfseries \@papername}}\hfill%  
\end{center}
\begin{center}
\vspace*{-5pt}
%\vspace*{0.1in}
\rule[0pt]{\textwidth}{0.5pt}\\
\makebox[0pt][l]{\small Ver.\MakeUppercase{\roman{volume}}  Rev.\arabic{issue}}\hfill
 \MakeUppercase{\small\it\@date}\hfill
\makebox[0pt][r]{\small\MakeUppercase{\@paperprice}}\\
\rule[6pt]{\textwidth}{1.2pt}
\end{center}
\pagestyle{plain}
}
\makeatother
%%%%%%%%%  Front matter   %%%%%%%%%%

\begin{document}
\maketitle

\begin{multicols}{3}

\byline{Diversity Statement}{Jacek Jo\'{n}ca-Jasi\'{n}ski}

\par I was born in the late 1970s to a working-class family in communist Poland. I was the first in my family to attend college and I ultimately moved abroad to pursue graduate education. Since Poland was unremarkably homogenous, it was not until arriving in the United States that I first had interacted with other races, ethnicities, religions, and people of different sexual orientations and gender identities. It was here that I worked minimum wage jobs, often hand-in-hand with members of other minority groups. Many of my co-workers became my friends and I credit this experience to my growing appreciation for other cultures. Pursuing graduate education in race and group relations further opened up my horizons. On one hand, I became immersed in a network of international students from five continents and developed friendships with members of the LGBTQx community. On the other hand, I explored academic literature on group relations, diversity and team performance, attitudes, race, and cognitive biases.
\par My own experiences as an immigrant made me uniquely attuned to the difficulties minority groups face. While working at Texas Tech University, I volunteered to assist students from disadvantaged populations, including students with disabilities, international students, and minorities. Perhaps the starkest example of a student who overcame multiple obstacles to pursue education was the case of a female, Latinx, first-generation college student with a congenital hearing impairment. The student has been caught amid an internal conflict within her academic department. The student made a strong case for an unequal pattern of treatment and had ample evidence to back it up. Unfortunately, when I met the student she no longer felt welcomed at the university and thought the personal cost of remaining with her graduate program was too high. Over the next two months, I helped her navigate the fragile political situation and together with the Dean of the Graduate School negotiated her departure from the university in good standing, and with an earned master's degree.
\par In other situations, my attempts to support diverse students yielded more positive outcomes. I assisted an Iranian student from a persecuted ethnic minority. The student had a documented psychological condition that worsened upon his arrival in the United States. When I first met him, he had difficulty answering basic questions and appeared fragile and visibly distressed. I walked him to the campus counseling center and made sure he received an emergency intake interview. Over the next two years, I continued to work with the student every week and coordinated with three other campus departments to provide him with the resources he needed to succeed. I observed incremental improvements in his happiness, cognitive functioning, and his academic performance and proudly watched him walk during his commencement ceremony.
\par Those and similar cases provided me with the examples necessary to push for broader institutional changes. These cases often illustrated systemic issues of inclusion and reflected a lack of will to walk an extra mile to accommodate those who are unlike one and might need extra attention. I worked with the Vice-Provost for Graduate and Postdoctoral Affairs and with the Interim Vice-President for Diversity, Equity, and Inclusion to change this culture, promote an inclusive campus climate, support diversity, gender equality, and to advocate on behalf of students with disabilities. I established and advised the graduate student association that engaged a diverse student population. Under my leadership, the association's executive officers included a U.S. Army veteran, several LGBTQx students, a first-generation college student, Black, Hispanic, and Asian students, as well as students of a half-dozen nationalities, including Saudi and Palestinian women. Drawing on volunteers from that group I opened and operated a food pantry providing supplementary nutrition to a predominantly international and underrepresented student populations. Together with the graduate association I also advocated on behalf of diversity at both the federal and state levels.
\par I am a living testimony that appreciation for differences is not innate and that many of us can and should develop it. Teamwork in diverse groups and intergroup contact are essential tools in helping to gain an appreciation for diversity. Nothing breaks down the boundaries better than jointly working towards a shared goal. Twenty years into this journey I find diversity to be a principle that underlies all beauty in nature. $\blacksquare$ % add the "tombstone" black square to the end of the text signaling end of article

%examples of using gothic initial with letterine package
%\lettrine[lines=3]{\color{BrickRed}T}{ he} package is basically a

%\lettrine[lines=3]{\gothfamily\fontsize{48pt}{50pt}\selectfont T}{ he} package is

%\lettrine[lines=2]{\gothfamily\selectfont T}{ he} package is basically

%\lettrine[lines=4]{T}{ he} package is basically a redefinition of the 

% Horizontal line newspaper article separator
%\closearticle

\end{multicols}

\end{document}